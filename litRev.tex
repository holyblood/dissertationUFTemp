\chapter{Literature Review}

\section{Empirical Likelihood}

Empirical likelihood is a nonparametric method of inference based
on data-driven likelihood ratio function. Like the bootstrap and jackknife,
empirical likelihood inference does not require us to specify a family
of distributions for the data. 
Also like parametric likelihood methods,
empirical likelihood makes an automatic determination of the shape
of confidence regions; it straightforwardly incorporates side information
expressed through constraints or prior distributions; it extends to
biased sampling and censored data, and it has very favorable asymptotic
power properties.

Empirical likelihood was first proposed by \citet{owen1988empirical}
as follows. For i.i.d sample $X_{1},\ldots,X_{n},$ the empirical
likelihood ration function is 
\[
R\left(\mu\right)=\max\left\{ \left.\prod_{i=1}^{n}nw_{i}\right|\sum_{i=1}^{n}w_{i}X_{i}=\mu,w_{i}\ge0,\sum_{i=1}^{n}w_{i}=1\right\} ,
\]
and the resulting asymptotic confidence region is 
\begin{equation}
\left\{ \mu\mid R\left(\mu\right)\ge\chi_{1,\alpha}^{2}\right\} =\left\{ \left.\sum_{i=1}^{n}w_{i}X_{i}\right|\prod_{i=1}^{n}nw_{i}\ge\chi_{1,\alpha}^{2},w_{i}\ge0,\sum_{i=1}^{n}w_{i}=1\right\} .\label{eq:ci-original-el}
\end{equation}
\begin{comment}
rubin 1981 bayesian bootstrap in wu changbao. ? lit review asymp exp
obj prior. 
\end{comment}
The following research generalizes this method to more complex settings,
\[
R\left(\theta\right)=\min\left\{ \left.d\left(\sum_{i=1}^{n}w_{i}I_{\left[x=X_{i}\right]},\sum_{i=1}^{n}\frac{1}{n}I_{\left[x=X_{i}\right]}\right)\right|\sum_{i=1}^{n}w_{i}g\left(X_{i},\theta\right)=0,w_{i}\ge0,\sum_{i=1}^{n}w_{i}=1\right\} ,
\]
where $g\left(\cdot,\cdot\right)$ is general estimating function,
and $d\left(\cdot,\cdot\right)$ is some divergence measure.%
\begin{comment}
prob cov ch2 ch3
\end{comment}
{} For more details, see \citet{owen2010empirical}.


\subsection{Different $g$ }

\citet{owen1991empirical} first considered normal equations
in linear regression as $g$. Later, almost all the facets in traditional
linear regression were transplanted into empirical likelihood. \citet{chen1993accuracy,chen1994empirical}
developed confidence regions for regression coefficients. \citet{jing1995two,adimari1995empirical}
considered the problem of comparing the means of two populations.
\citet{davidian1987variance} investigated different variance structures
in regression. \citet{kolaczyk1995information} formulated empirical
information criterion as an empirical likelihood version of AIC. 

Besides linear regression, the authors in this area also took efforts
to incorporate more complex models into empirical likelihood. \citet{chen1993smoothed}
used kernel smoothing technique to construct $g$ for quantiles. \citet{whang2006smoothed}
extended this technique into empirical likelihood quantile regression.
\citet{kolaczyk1994empirical} further extended $g$ into generalized
linear model. \citet{hall1993empirical} studied empirical likelihood
confidence bands for kernel density estimates. \citet{chen2000empirical}
studied empirical likelihood confidence intervals for local linear
kernel smoothing. \citet{peng2004empirical} investigated empirical
likelihood confidence intervals for heavy-tailed distributions.

Data drawn from sophisticated designs have also been taken into consideration.
\citet{qin1993empirical} applied empirical likelihood to biased sampling
data, whose distribution $G$ related the distribution of interest
$F$ through a biasing function $u$, that is
\[
G\left(A\right)=\frac{\int_{A}u\left(y\right)\diff F\left(y\right)}{\int u\left(y\right)\diff F\left(y\right)},
\]
for every Borel set $A$.
This result was further generalized by \citet{qin1997goodness,qin1999empirical} and \citet{qin1998semiparametric}
for biased sampling. \citet{chen1993empirical}
presented empirical likelihood for samples from a finite population.
\citet{chen1999pseudo} formulated an empirical likelihood that incorporated
design weights. \citet{wu2001model} considered a setting where the
entire population is known but only the sample are available. \citet{loh1996latin}
investigated Latin hypercube samples by empirical likelihood confidence
region. Empirical likelihood analysis of cumulative hazard function
was established by \citet{murphy1995likelihood}. \citet{adimari1997empirical}
considered empirical likelihood inferences for the mean under independent
right censoring and later \citet{pan2002empirical} extended it to
do inference about more general functionals of the hazard function
$\int q_{n}\left(x\right)\diff\Lambda\left(x\right)$. \citet{murphy1997semiparametric}
considered general double censoring and proportional hazard
model.

Inference on infinite dimensional parameters, such as CDF and quantile
functions are also visited. \citet{owen1995nonparametric} built exact
confidence bands for CDF based on empirical likelihood. \citet{hollander1997likelihood}
found asymptotic confidence bands for survival function for right
censoring data. \citet{zhang1996confidence,zhang1999bootstrapping}
constructed confidence bands when auxiliary information was available. 

Finally, for independent data, I refer to \citet{qin1994empirical}
as a useful general case. They analyzed the behavior of empirical
likelihood when $g$ was a smooth estimating function and number of
parameters was less than number of constraint equations. They proposed
maximum empirical likelihood estimators which could be viewed as
an alternative to least square estimators when the data could not square
with the model. \citet{molanes2009empirical} extended this work into
non-smooth criterion function. 

Dependent data can also be analyzed by empirical likelihood. \citet{chuang2002empirical}
applied empirical likelihood to unstable auto-regression. \citet{kitamura1997empirical}
developed block-wise empirical likelihood for weakly dependent data
and extended the result in \citet{qin1994empirical}. \citet{chen2009smoothed}
combined block-wise and smooth techniques in order to estimate quantiles
for weakly dependent data. \citet{nordman2007empirical} applied block-wise
empirical likelihood for long-range dependence processes but the limit
distribution became a multiple Wiener integral instead of simple $\chi^{2}$.
The spectral approach to empirical likelihood was due to \citet{monti1997empirical}.
\citet{mykland1995dual} extended empirical likelihood into martingale
by dual likelihood technique. Suppose $m_{n}\left(\theta\right)$
is a zero mean martingale depending on data and model, for example,
a sample estimating equation or martingale leading to Nelson--Aalen
estimator, then the log dual likelihood $l\left(\mu\right)$ for dual
parameter $\mu$ and fixed $\theta$ is defined by partial differential
equations 
\[
m_{n}\left(\theta\right)=\left.\nabla l_{\theta}\left(\mu\right)\right|_{\mu=0},
\]
 or through Dol\textipa{\'e}ans--Dade multiplicative martingale 
\[
l_{\theta}\left(\mu\right)=-\mu^{T}\Lambda_{t}\left(\theta\right)+\sum_{s\le t}\left(1+\mu^{T}\Delta m_{s}\left(\theta\right)\right).
\]
This definition coincides empirical likelihood for i.i.d data and discrete time. \citet{wang2011empirical} addressed quantile
and the corresponding likelihood ratio test is $\mathrm{LR}=2\sup_{\mu}l_{\theta}\left(\mu\right)$.
regression with longitudinal data by empirical likelihood. A more
recent study by \citet{bandyopadhyay2015frequency} introduced empirical
likelihood into spatial data analysis. 


\subsection{Different $d$ }

There is also literature on the effect of different empirical divergence
measures in empirical likelihood settings. \citet{hall1999biased}
used empirical entropy to construct robust confidence regions for
non-robust statistics like the mean. \citet{baggerly1998empirical}
extended $d$ to Cressie-Read family in the sample mean case and proved
empirical likelihood was the only Bartlett correctable member of that
family. In econometrics, \citet{mittelhammer2000econometric} used
empirical entropy and Cressie--Read divergence measure in $d$ and
linked maximum empirical likelihood estimator to general method of
moment. To overcome the drawback that empirical likelihood domain
$\Theta$ was usually smaller than the full parameter space $\mathbb{R}^{p}$,
\citet{tsao2014extended} defined a surjective composite similarity
mapping 
\[
h^{C}\left(\theta\right)=\hat{\theta}_{\mathrm{MELE}}+\left(1+\frac{R\left(\theta\right)}{2n}\right)\left(\theta-\hat{\theta}_{\mathrm{MELE}}\right),
\]
from $\Theta$ to $\mathbb{R}^{p}$, and used its generalized inverse
to adjust original empirical likelihood into $R\left(\left(h^{C}\right)^{-1}\left(\theta\right)\right)$. 


\subsection{Higher-Order Properties }

Higher order properties have also been investigated. Bartlett
correctable is one of the most favorite aspects of using empirical likelihood.
This correction is usually based on an Edgeworth expansion of the empirical
likelihood version, and has the following form: instead of using \ref{eq:ci-original-el},
the more accurate version is 
\[
\left\{ \theta\mid-2\log R\left(\theta\right)\le\left(1+\frac{a}{n}\right)\chi_{p,1-\alpha}^{2}\right\} ,
\]
or 
\[
\left\{ \theta\mid-2\log R\left(\theta\right)\le\left(1-\frac{a}{n}\right)^{-1}\chi_{p,1-\alpha}^{2}\right\} .
\]
The fist Bartlett correction was given in \citet{diciccio1991empirical}
for the sample mean. \citet{zhang1996accuracy} extended it to general
estimating function. However, \citet{lazar1999empirical} reported
failure of Bartlett correction when there were nuisance parameters.
This is where the empirical likelihood shows different
behavior from parametric likelihoods. This problem was solved by \citet{chen2006bartlett}.
Other Bartlett correction usually follows the development of the empirical
likelihood model. \citet{biao1998note} showed that there was no first
order asymptotic benefit from global side constraints in kernel density
estimation. However, \citet{chen1997empirical} proved that the second
order asymptotic benefit could be expected via imposing side information.
\citet{newey2004higher} considered higher order properties of
maximum empirical likelihood estimators in the general method of moments
setting. \citet{kitamura2001asymptotic} gave some theorems on the
large deviation properties of empirical likelihood, and his simulation
suggested empirical likelihood performed very well at hypotheses father
from the null. 


\subsection{High Dimension and Sparsity}

High dimensional data and sparsity are pivotal topics in recent years
and they also shed light on empirical likelihood. \citet{tsao2004bounds}
pointed out the failure of empirical likelihood even if the number
of parameters was moderately larger than the number of observations $p/n>1/2$.
\citet{hjort2009extending} rescaled the empirical likelihood itself
by number of parameters and asserted a normal limit distribution. Their
work was generalized by \citet{chen2009effects} with less restrictions.
\citet{chen2008adjusted} and \citet{emerson2009calibration} proposed adjusted
empirical likelihood by adding pseudo observations. \citet{bartolucci2007penalized}
added Tikhonov regularization defined sample covariance matrix and
succeeded when growing $p<n$. \citet{lahiri2012penalized} added
component wise penalties and extended the result into weak and  long-range
dependency and $p>n$. \citet{tang2010penalized} added popular SCAD
penalty to traditional empirical likelihood for population mean and
validated sparsity in high dimensional settings. \citet{leng2012penalized}
extended this work to generalized estimating equations and more general
penalties. An interesting result from \citet{chang2015high} shows
the usual block-wise empirical likelihood is still working if the
number of constraints is growing with the number of parameters in some
rate.


\subsection{Bayesian }

Compared to the fruitful frequentist research in empirical likelihood,
Bayesian counterparts are just at the beginning. \citet{lazar2003bayesian}
started the Bayesian empirical likelihood and did many simulations
to prove its validation. \citet{schennach2005bayesian,schennach2007point}
explained the Bayesian exponentially tilted empirical likelihood as
the limit of some nonparametric procedure. \citet{grendar2009asymptotic}
found the equivalence between maximum empirical likelihood estimators
and Bayesian maximum a posteriori probability estimators under model
misspecification. \citet{lancaster2010bayesian} used Bayesian exponentially
tilted empirical likelihood for quantile regression. \citet{fang2005expected,fang2006empirical}
established probability matching prior for empirical likelihood for
population mean case, which allowed the credible intervals to have
validity in the frequentist sense. \citet{chang2008bayesian} consolidated
this result by showing empirical likelihood was the only member who
enjoyed both Bayesian and frequentist validity. \citet{vexler2013nonparametric}
used empirical likelihood as a non-parametric method to estimate Bayes
factor in model selection. \citet{vexler2014posterior} showed some
James-Stein phenomenon for Bayesian empirical likelihood. \citet{rao2010bayesian}
 a prior on empirical
weights $w_{i}$ in complex survey settings.

\begin{comment}
more papers in abc folder, pro and con of el
\end{comment}



\section{Option Pricing}

Since the seminal works by \citet{black1973pricing} and \citet{merton1973intertemporal},
option valuation methodologies have been extensively developed. The
Black-Scholes model has become one of the most well-known discoveries
in finance literature, which relates the cross-sectional properties
of option prices with the underlying asset return distribution.
However, \citet{rubinstein1985nonparametric} and \citet{melino1990pricing} pointed out
several limitations in the Black-Scholes model due to the strong assumptions,
such as non-normality of the returns, stochastic volatility (implied
volatility smile), jumps and others. Both parametric and nonparametric
approaches have been proposed to deal with these issues. 

\citet{scott1987option,hull1987pricing} and \citet{wiggins1987option} extended the
Black and Scholes model and allowed the volatility to be stochastic.
\citet{heston1993closed} developed a closed-form solution for option pricing
with the underlying assets volatility being stochastic. \citet{duan1995garch}
proposed a GARCH option pricing model in an attempt to explain some
systematic biases associated with the Black-Scholes model. Later \citet{heston2000closed} provided a closed-form solution for option pricing
with the underlying assets volatility following GARCH$(p,q)$ process.
\citet{bates1996jumps} and  \citet{bakshi1997empirical} derived an option pricing
model with stochastic volatility and jumps. \citet{kou2002jump} provided a
solution to pricing the option with the double exponential jumps diffusion
process. \citet{carr1999option} introduced the fast Fourier transform
approach to option pricing given a specified characteristic function
of the return, which provides an efficient computational algorithm
to calculate the option prices. For further reference, see \citet{duffie2000transform,bakshi2000spanning},  and \citet{carr2009saddlepoint} among
others. All these methods are parametric based, which assume a parametric
form of either the distribution of the underlying assets returns or
the characteristic function of the underlying assets returns. 

Nonparametric approaches have also been proposed to capture the underlying
asset and option price data to reconstruct the structure of the diffusion
process. For example, \citet{hutchinson1994nonparametric} applied the
neural network techniques to price the derivatives. \citet{ait1998nonparametric} used the kernel regression to fit the state-price density
implicitly in option pricing. \citet{ait1996nonparametric} proposed a nonparametric
pricing estimation procedure for interest rate derivative securities
under the assumption that the unknown volatility is independent of
time. \citet{stutzer1996simple} adopted the canonical valuation method, which
incorporats the no-arbitrary principle embodied in the formula for
calculating the expectation of the discounted value of assets under
the risk-neutral probability distribution. %




\section{Approximate Bayesian Computation}

Approximate Bayesian Computation (ABC) methods, also known as likelihood-free
techniques, have appeared in the past ten years as the most satisfactory
approach to intractable likelihood problems, first in genetics then
in a broader spectrum of applications. Intractable likelihood is a
common phenomenon in statistical modeling. 
\begin{itemize}
\item The likelihood is expressed as a multidimensional integral, 
\[
l\left(\theta\mid Y\right)=\int l^{*}\left(\theta\mid Y,u\right)\diff u,
\]
where $Y$ is observation, $u$ is latent variable and $\theta$ is
the parameter of interest, for example, coalecent model in population
genetics. Typically when the dimension of $u$ is large, the convergence
properties of MCMC like Gibbs sampler and Metropolis--Hastings algorithm
are too poor to use in practice. 
\item The normalizing constant is unknown. This is typically the case of
Gibbs random fields in order to model spatially correlated data such
as epidemiology and image analysis. 
\item The likelihood function is not completely known, that is 
\[
l\left(\theta\mid Y\right)=l_{1}\left(\theta\mid Y\right)l_{2}\left(\theta\right).
\]
In the past, Laplace approximations by \citet{tierney1986accurate}
or variational Bayes solutions by \citet{jaakkola2000bayesian} have
been advanced for such problems. However, Laplace approximations require
some analytic knowledge of the posterior distribution, while variational
Bayes solutions replace the true model with another pseudo-model which
is usually much simpler and thus misses some of the features of the
original model.
\end{itemize}
\begin{comment}
abc machine learning
\end{comment}


The idea of ABC dated back to \citet{rubin1984bayesianly} as an intuitive
way to understand posterior distributions from a frequentist's perspective,
because parameters from the posterior are more likely to be those
that could have generated the observed data. The first ABC algorithm
was born in \citet{tavare1997inferring} and  \citet{pritchard1999population}
as 
\begin{algorithm}[h]
\begin{enumerate}
\item Sample parameters $\theta_{i}$ from the prior distribution $\pi\left(\theta\right)$;
\item Sample data $Z_{i}$ based on the model $f\left(z\mid\theta_{i}\right)$;
\item Accept $\theta_{i}$ if $\rho\left(S\left(Z_{i}\right),S\left(X_{\mathrm{obs}}\right)\right)\le\varepsilon$, {{}
} {for some metric $\rho$ and summary statistics
$S$. }
\end{enumerate}
\protect\caption{\label{alg:Prichard-ABC}Prichard's Modified ABC}
\end{algorithm}
For more details, see \citet{marin2012approximate}.


\subsection{Different Sampling Schemes}

In practice, if one uses non-informative prior, simulation would be
very inefficient, because of high rejection rate of prior sample locating
in low posterior probability regions. As an answer to this problem,
\citet{marjoram2003markov} applied Metropolis--Hastings algorithm
in sampling from prior distributions and built MCMC-ABC algorithm.
\citet{picchini2014inference} used this method to analyze data from
stochastic differential equations. \citet{lee2014variance} backed
up this method by several theoretical results including variance bound
and geometric ergodicity. \citet{ratmann2009model} used tolerance
level $\varepsilon=\rho\left(S_{i},S_{\mathrm{obs}}\right)$ as an
additional parameter of the model and proposed ABC$_{\mu}$, as method
to assess model uncertainty. \citet{wilkinson2013approximate} replaced
the hard accept-reject scheme by soft kernel smoothing, called noisy
ABC, 
\[
\pi_{\varepsilon}\left(\theta,z\mid Y\right)\propto\pi\left(\theta\right)f\left(z\mid\theta\right)K_{\varepsilon}\left(Y-z\right),
\]
where $K_{\varepsilon}$ is a well-chosen kernel parameterized by
the bandwidth $\varepsilon$. He also made the valuable point that
noisy ABC simulated exactly from the posterior conditioning on observations
with errors. \citet{sisson2007sequential} combine partial rejection
control and ABC to solve the inefficiency when prior and posterior
are dissimilar. In order to analyze hidden Markov models, \citet{jasra2012filtering}
proposed ABC filtering incorporated sequential Monte Carlo (SMC) method.
This procedure was theoretically justified in parameter estimation by
\citet{dean2014parameter}. %
\begin{comment}
detail of justification
\end{comment}
Using SMC sampler will result in a bias in approximation to the posterior.
To overcome this problem, \citet{beaumont2009adaptive} incorporated
population Monte Carlo method into ABC-PRC by modifying the importance
sampling weights with an component-wise random walk estimator of likelihood
and using decreasing tolerance levels. MCMC-ABC also suffers from
poor mixture properties when tolerance level is small. To rectify this
weakness, \citet{baragatti2013likelihood} proposed ABC parallel tempering
scheme. The basic technique is running several MCMC-ABC chains with
different tolerance levels and swap some chains under certain conditions.
They recommended ABC-PT when the posterior was multi-modal. 


\subsection{Calibration}

\citet{mckinley2009inference} performed a simulation comparing ABC-MCMC
and ABC-SMC. The conclusions are the choice of the distance, the summary
statistics are paramount to the success of ABC, while the tolerance
level does not seem to have a strong influence.

For parameter estimation, the ideal summary statistics would be the
sufficient statistics. However, for most real problems, it is impossible
to find them. \citet{joyce2008approximately} considered sequential
inclusion of summary statistics based on likelihood ratios. Nevertheless,
their method does not address the issue of construction of summary
statistics and does not take into account the sequential nature of
likelihood ratios. \citet{aeschbacher2012novel} advocated inclusion
via boosting. \citet{blum2013comparative} summarized several other
methods to select summary statistics, such as information criterion,
partial least square regression, neural network. \citet{fearnhead2012constructing}
used polynomial regression to estimate the posterior mean and used
it as a summary statistic. To our knowledge, this is the
first method which is able to  construct automatic summary statistics.
\citet{ruli2013approximate} suggested using score function of composite
likelihood as summary statistics. \citet{barthelme2014expectation}
applied expectation propagation approximation in ABC, that is approximating
the posterior by 
\[
\pi\left(\theta\right)\prod_{i=1}^{n}\int f\left(Z_{i}\mid Y_{1},\ldots,Y_{i-1},\theta\right)I_{\left[\left|S_{i}\left(Z_{i}\right)-S_{i}\left(Y_{i}\right)\right|\le\varepsilon\right]}\diff Z_{i},
\]
where $S_{i}$ is a local summary statistics for a lower dimensional
data $Z_{i}$, typically $S_{i}\left(Z_{i}\right)=Z_{i}$. By replacing
global summary statistics with local ones, they proved their algorithm
EP-ABC was faster than usual ABC because the accept rate might be
higher.

For model selection, the situation is complex. \citet{grelaud2009abc}
used sufficient statistics in model selection between Gibbs random
fields. However, \citet{marin2014relevant} suggested the statistics
auxiliary under all candidate models as the best summary statistics
in model selection based on Bayesian factor.%
\begin{comment}
add the paper post the problem of model selection in abc
\end{comment}


Calibration of tolerance levels has also attracted many attention. \citet{biau2012new}
viewed the rejection based on metric as a $k$-nearest neighbor procedure
and then calibration on $\varepsilon$ is equivalent to calibration
of $k$. Their results favor $k\approx N^{\left(p+4\right)/\left(p+d+4\right)}$,
where $N$ is the number of simulations from model, $p$ is the dimension
of the parameters of interest, and $d>4$ is the dimension of the
summary statistics, and under this calibration, they derived rate
of convergence of mean square error of density estimation. \citet{ratmann2013statistical}
treated the calibration of $\varepsilon$ as a statistical hypothesis
testing problem and obtained $\varepsilon$ as a critique value in
hypothesis testing. The advantage of their method is the MAP estimate
is the same under full posterior and ABC posterior and the Kullback--Leibler
divergence of the two distributions is small. 


\subsection{Post-Process}

Besides modifying sampling scheme and summary statistics, one could
also improve inference by carefully processing the output from
ABC. Viewing approximation to the posterior as a conditional density
estimation problem, \citet{beaumont2002approximate} applied local
linear regression by replacing the simulated raw $\theta$ by 
\[
\theta^{*}=\theta-\left(S\left(z\right)-S\left(Y\right)\right)^{T}\hat{\beta},
\]
where $\hat{\beta}$ is obtained by a weighted least square regression,
using weights of the form $K_{\varepsilon}\left(\rho\left(S\left(z\right),S\left(Y\right)\right)\right)$.
\citet{blum2010non} generalized this idea to nonlinear regression
with heteroskedasticity estimated by a neural net with one hidden
layer. \citet{leuenberger2010bayesian} addressed the same issue using
inverse regression.


\section{Sufficient Dimension Reduction}



Sufficient dimension reduction (SDR) is an emerging topic in recent
statistical area. As one of the answers to high dimension problems
, SDR usually has solid theoretical background to guarantee large
sample consistency and valid statistical procedures to select the
dimension of results, comparing with machine learning techniques such
as manifold learning. The original problem is formulated as follows.
Let $X\in\mathbb{R}^{p}$ and $Y\in\mathbb{R}$, and there is a unknown
lower dimensional transformation $S:\mathbb{R}^{p}\rightarrow\mathbb{R}^{d}$,
where $d<p$, the one we need to estimate, which satisfies 
\[
P\left(Y\le y\mid X\right)=P\left(Y\le y\mid S\left(X\right)\right),\forall y\in \mathbb{R}.
\]
Most papers in SDR focus on the case where  $S$ is a
linear transformation, that is $S\left(X\right)=\beta^{T}X$, where
$\beta\in\mathbb{R}^{p\times d}$. Note that even in linear SDR, 
inference differs from transitional parameter estimation, because
for any non-singular matrix $T$, $\beta T$ is still a valid dimension
reduction transformation. As a result, the exact values of each
entries in $\beta$ is not identifiable, but the column space of $\beta$
is uniquely determined. So the parameters in linear SDR should essentially
be in a space called central subspace denoted by $S_{Y\mid X}$ in \citet{cook1994interpretation}
and the optimization problems are in general
constrained in the set of subspaces called Grassmann manifold instead
of the usual Euclidean space. %
\begin{comment}
non-linear dr by cond on sigma algebra
\end{comment}
Sometimes, we are only interested in $E\left(Y\mid X\right)$, for
example, in linear regression. Then a weak assumption can be made
as 
\[
E\left(Y\mid X\right)=E\left(Y\mid\beta^{T}X\right).
\]
The corresponding space of $\beta$ is called central mean subspace
$S_{E\left(Y\mid X\right)}$ in \citet{cook2002dimension}. \citet{yin2002dimension}
generalized the idea into central $k$-th moment subspace $S_{Y\mid X}^{\left(k\right)}$
defined as 
\[
E\left(Y^{j}\mid X\right)=E\left(Y^{j}\mid\beta^{T}X\right),\mathrm{\: for\:}j=1,\ldots,k.
\]
 To estimate the conditional variance, \citet{zhu2009dimension} introduced
the notion of central variance subspace $S_{\Var\left(Y\mid X\right)}$,
defined as 
\[
\Var\left(Y\mid X\right)=E\left(\Var\left(Y\mid X\right)\mid\beta^{T}X\right).
\]


To estimate the defined space, statisticians innovated several methods,
which can be roughly classified into three categories: inverse regression
methods, non-parametric methods, and semi-parametric methods. For
simplicity, they usually assume $X$ has zero mean and identity variance-covariance
matrix. \citet{lee2013general} formulated the ideas by general measure
theory and central $\sigma$-field $\mathscr{G}_{Y\mid X}$ as 
\[
Y\independent X\mid\mathscr{G}_{Y\mid X}.
\]
 For more details, I refer to \citet{ma2013review}.


\subsection{Inverse Regression }

The first inverse regression method, sliced inverse regression (SIR)
proposed by  \citet{li1991sliced}, is a precursor of SDR. In that
paper, he proved 
\[
E\left(X\mid Y\right)\in S_{Y\mid X},
\]
and used principal component analysis (PCA) to get the main direction
of several estimators $\hat{E}\left(X\mid Y=y_{1}\right),\ldots,\hat{E}\left(X\mid Y=y_{s}\right)$
by sliced mean. This paper also built an exemplary approach for other inverse
regression based methods. He showed that key quantities, mostly
conditional moments, belonged to the corresponding central space, then
use PCA to find the main direction of these key quantities. This paper
also invents two standard conditions in linear SDR, named linearity
condition 
\[
E\left(X\mid\beta^{T}X\right)=L\beta^{T}X,
\]
and constant covariance condition 
\[
\Var\left(X\mid\beta^{T}X\right)=Q,
\]
where $Q$ is a non-random matrix. The two condition restrict the
usage of SIR to nearly normal $X$. Lately, \citet{dong2010dimension}
relaxed the linearity condition to polynomial condition, that is,
$E\left(X\mid\beta^{T}X\right)$ is a polynomial function of $\beta^{T}X$.
Inspired by SIR, \citet{zhu1996asymptotics} proposed kernel inverse
regression using kernel technique to estimate the same key quantities
in SIR. \citet{wu2008kernel} generalized this to nonlinear SDR problem.
\citet{wang2014transformed} applied probability integral transformation
to SIR in order to solve nonlinear SDR. \citet{li2011principal} replaced
moments key quantities by a more robust quantile-like quantities defined
by a sliced SVM. They proved $\psi\left(y\right)\in S_{Y\mid X}$,
if $\psi$ was a solution of a generalized SVM problem, 
\[
\left(\psi\left(y\right),t\left(y\right)\right)=\argmin_{\psi,t}\psi^{T}\hat{\Sigma}_{X}\psi+\lambda E_{X,Y}\left(1-\left(I_{\left[Y\le y\right]}-I_{\left[Y>y\right]}\right)\left[\psi^{T}\left(X-\overline{X}\right)-t\right]\right)_{+}.
\]
Their method, called principal support vector machine, can be applied
to both linear and kernelized nonlinear SDR problem.As pointed in
\citet{1991}, SIR fails when there are some symmetric patterns, ,
so they proposed sliced average variance estimation (SAVE) using the
second conditional moments, 
\[
\mathrm{span}\left(I_{p}-\Var\left(X\mid Y\right)\right)\subset S_{Y\mid X}.
\]
\citet{zhu2007hybrid} suggested a hybrid of SIR and SAVE by a convex
combination. \citet{li2007directional} proposed direction regression
(DR) based on 
\[
\mathrm{span}\left(2I_{p}-E\left(\left.\left(X-\tilde{X}\right)\left(X-\tilde{X}\right)^{T}\right|Y,\tilde{Y}\right)\right)\subset S_{Y\mid X}.
\]
To avoid tuning parameters such as the number of slices and bandwidth
of kernel, \citet{zhu2010sufficient} proposed discretization-expectation
procedure. \citet{zhu2010dimension} proposed cumulative slicing estimation
and \citet{li2005contour} proposed contour regression. The above
methods usually use non-parametric or semi-parametric method to estimate
the key quantities, to take the advantage of explicit likelihood,
\citet{cook2009likelihood} introduced likelihood acquired directions
as MLE of inverse regression. 

For central mean subspace, \citet{li1989regression} proved the column
space of ordinary least squares is a subspace of $S_{E\left(Y\mid X\right)}$.
\citet{li1992principal} proposed principal Hessian directions and
\citet{cook2002dimension} proposed iterative Hessian transformations
to recover $S_{E\left(Y\mid X\right)}$. 

Multivariate response settings are also considered. %
\begin{comment}
add envelope and dr ask dr su 
\end{comment}
Three main stream methods are developed. The first is generalizing
sliceing into hypercubes defined by different topologies, for instance,
\citet{aragon1997gauss,hsing1999nearest,setodji2004k}.The second
is recovering the joint central subspace from marginal central subspaces,
for instance, \citet{cook2003model,saracco2005asymptotics,yin2006moment}.
The last one is projecting multivariate response onto lower dimensional
space, for instance, \citet{li2008projective,zhu2010dimension}. 


\subsection{Non-Parametric Methods}

Non-parametric methods do not require the linearity condition or constant
covariance condition. And thus these methods are more flexible than
inverse regression methods. However, they still rely on continuous
$X$, and hence could not be applied to categorical predictors.

The first non-parametric method is the minimum average variance estimation
(MAVE) by \citet{xia2002adaptive}, which estimated the central mean
subspace by kernel weighted least square subject to Grassmann manifold
restriction. The advantage of this method is exhaustiveness, meaning
that it would recover the whole $S_{E\left(Y\mid X\right)}$ if $d$
is correctly specified. %
\begin{comment}
add bayesian method khare
\end{comment}
This method settles the characters of almost all non-parametric methods,
that is, smoothing approach to unknown link function $m$ defined
as 
\[
Y=m\left(\beta^{T}X\right)+\varepsilon.
\]
Lately, \citet{xia2007constructive} proposed density based MAVE to
estimate central subspace, basically replacing response $Y$ by kernel
smoothing $K_{b}\left(Y-y\right)$, and \citet{wang2008sliced} proposed
sliced regression. \citet{hernandez2005dimension,yin2005direction,yin2008successive}
replaced the weighted least square by other loss functions.


\subsection{Semi-Parametric Methods}

As far as the authors know, there is only one semi-parametric method
by \citet{ma2012semiparametric} now. In the same way as in sufficient
statistics , they decomposed the likelihood functions into two parts,
one of which contained only predictors, and the other of which contained
response, and the only interest in dimension reduction community was
the latter part. Based on this observation, they formulated consistency
estimating equations by influence function class defined as 
\[
\left\{ f\left(Y,X\right)-E\left(f\left(Y,X\right)\mid\beta^{T}X,Y\right):E\left(f\left(Y,X\right)\mid X\right)=E\left(f\left(Y,X\right)\mid\beta^{T}X\right),\forall f\right\} ,
\]
particularly, the following forms were used 
\[
f\left(Y,X\right)=\left(g\left(Y,\beta^{T}X\right)-E\left(g\left(Y,\beta^{T}\right)\mid\beta^{T}X\right)\right)\left(\alpha\left(X\right)-E\left(\alpha\left(X\right)\mid\beta^{T}X\right)\right),
\]
for any $g$ and $\alpha$. This approach does not require moment
conditions like linearity condition or constant covariance condition,
or continuous condition of predictors. The authors also shew in their
paper several inverse regression methods could be derived by different
settings of $g$ and $\alpha,$ and the statistical intuitive of linearity
condition and constant covariance condition could also be explained
under this framework.


\subsection{Inference about $d$}

Statistical inference about reduction dimension is a characteristic
of sufficient dimension reduction. There are sequential test methods
by \citet{schott1994determining,velilla1998assessing,bura2001extending,cook2001special,cook2004determining,cook2005sufficient}.
\citet{ye2003using} and \citet{zhu2006fourier} proposed bootstrap
method to select $d$. BIC-type methods are considered in \citet{zhu2006sliced,zhu2007kernel,luo2009contour}.
A more fashion method is proposed by \citet{zhu2010sparse} as sparse
eigen-decomposition strategy. They imposed adaptive LASSO penalty
to spectral decomposition problem in inverse regression, and the minimization
could be solved very effectively by LARS algorithm in \citet{efron2004least}. 


\subsection{High Dimension}

The laurel of modern statistics should be crowned to inference under
increasing number of parameters. To avoid singularity of marginal
covariance matrix of $X$ when $p>n$, some authors incorporate partial
least squares into inverse regression, such as in \citet{li2007partial,cook2007dimension,zhu2009distribution,zhu2010dimension}.
Another strategy is to utilize the sparsity principle, such as LASSO,
SCAD and Dantzig selector. This leads to sure independence ranking
and screening procedure in \citet{zhu2011model}. \citet{wu2008consistency}
applied Tikhonov penalties to kernel sliced inverse regression to
solve nonlinear SDR. %
\begin{comment}
add wuchang bao setting consistency and normalty proof
\end{comment}


