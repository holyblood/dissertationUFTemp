\chapter{Proof for Higher-Order Properties of Bayesian Empirical Likelihood:
Univariate Case}

\section{Behavior Of Log Empirical Likelihood In  The Tail}

The Taylor expansion consists of  expanding the log empirical
likelihood and prior density around the mean and  then control
the tail part of the log empirical likelihood. In order to implement
this idea, the tail part of $\tilde{l}\left(\theta\right)$ must vanish
faster than the required polynomial order. In this section,
we will show that indeed the tail part of $\tilde{l}\left(\theta\right)$
vanishes at an exponential rate.
\begin{lemma}
\label{lemma:exponential-decay-tail} Under  Assumptions 1 and 2, for any $\delta_{1}>0$, there
exist $\varepsilon_{1}>0$ and $N_{3}$, such that 
\[
\tilde{l}\left(\theta\right)-\tilde{l}\left(\tilde{\theta}\right)\le-\varepsilon,\ascv,
\]
for any $\left|b\left(\theta-\tilde{\theta}\right)\right|\ge\delta_{1}$
and $\theta\in H_n$, where $b=\left[\left\{ n^{-1}\sum_{i=1}^{n}\diff g\left(X_{i},\tilde{\theta}\right)/\diff\theta\right\} ^{2}/\left\{ n^{-1}\sum_{i=1}^{n}g\left(X_{i},\tilde{\theta}\right)^{2}\right\} \right]^{-1/2}$\end{lemma}
\begin{proof}
By Assumption 6, we know that $\tilde{\theta}$ is the unique maximizer%
\begin{comment}
need to be assume
\end{comment}
. Therefore, for any $\theta\neq\tilde{\theta}$, $\tilde{l}\left(\theta\right)<\tilde{l}\left(\tilde{\theta}\right)$.
The set 
\[
\left\{ \theta:\left|b\left(\theta-\tilde{\theta}\right)\right|\ge\delta_{1}\right\} \cap H_n
\]
is a compact set, and $\tilde{l}\left(\theta\right)$ is a continuous
function. Hence there exists a $\theta^{*}\in\left\{ \theta:\left|b\left(\theta-\tilde{\theta}\right)\right|\ge\delta_{1}\right\} \cap H_n$
, such that for any $\theta\in\left\{ \theta:\left|b\left(\theta-\tilde{\theta}\right)\right|\ge\delta_{1}\right\} \cap H_n$,
\[
\tilde{l}\left(\theta\right)\le\tilde{l}\left(\theta^{*}\right).
\]
Therefore, $\tilde{l}\left(\theta\right)\le\tilde{l}\left(\theta^{*}\right)<\tilde{l}\left(\tilde{\theta}\right)$, which is equivalent to 
\[
\tilde{l}\left(\theta\right)-\tilde{l}\left(\tilde{\theta}\right)\le\tilde{l}\left(\theta^{*}\right)-\tilde{l}\left(\tilde{\theta}\right)<0.
\]
Let $\varepsilon_{1}=\left\{ \tilde{l}\left(\tilde{\theta}\right)-\tilde{l}\left(\theta^{*}\right)\right\} /2$,
then we have 
\[
\tilde{l}\left(\theta\right)-\tilde{l}\left(\tilde{\theta}\right)\le\tilde{l}\left(\theta^{*}\right)-\tilde{l}\left(\tilde{\theta}\right)<\varepsilon_{1}.
\]

\end{proof}

\section{Higher-Order Derivatives}\label{app:high-order-der}

In order to expand around the mean, we need to control the
remainder terms in the Taylor expansion, which involves the finiteness of higher-order
derivatives of $\tilde{l}$. 
\begin{lemma}
\label{lem:control-higher-order-derivative-l}Let%
\begin{comment}
this lemma need change
\end{comment}
{} 
\[
\omega_{i}\left(\theta\right)=\begin{cases}
\left\{ 1+\nu\left(\theta\right)g\left(X_{i},\theta\right)\right\} ^{-1}, & \text{for empirical likelihood,}\\
\exp\left\{ -\nu\left(\theta\right)g\left(X_{i},\theta\right)\right\} , & \text{for exponentially tilted empirical likelihood,}\\
\left\{ \mu\left(\theta\right)+\nu\left(\theta\right)g\left(X_{i},\theta\right)\right\} ^{-1/\left(\lambda+1\right)}, & \text{for Cressie-Read empirical likelihood.}
\end{cases}
\]
and 
\[
D=\begin{cases}
n^{-1}\sum_{i=1}^{n}\omega_{i}^{r}g\left(X_{i},\theta\right)^{2}, & \text{for empirical and exponentiallytilted,}\\
n^{-1}\sum_{i=1}^{n}\omega_{i}^{\lambda+2}\left[g\left(X_{i},\theta\right)-\left\{ n^{-1}\sum_{i=1}^{n}\omega_{i}^{\lambda+2}g\left(X_{i},\theta\right)\right\} \right]^{2}, & \text{for Cressie-Read.}
\end{cases}
\]
Then  under Assumptions 1 and 3, for any $k=2,\ldots K+3$ , 
\begin{eqnarray*}
\frac{\diff^{k}}{\diff\theta^{k}}\tilde{l}\left(\theta\right) & = & D^{-r_{k}}P_{k}\left(M_{1},M_{2},\ldots,\nu,\mu\right),
\end{eqnarray*}
where $P_{k}$ are polynomial functions , and all the $r_{j}<C\left(k\right)$,
and $C\left(k\right)$ is some constant only depending on $k$, and
$M_{j}$ are the weighted average of higher order derivatives of $g$,
i.e. 
\[
M_{j}=\frac{1}{n}\sum_{i=1}^{n}\omega_{i}^{r}\prod_{l}\frac{\diff^{l}g\left(X_{i},\theta\right)}{\diff\theta^{l}},
\]
$l$ can be the same . \end{lemma}
\begin{proof}
From Lemma 2, we know that 
\[
\frac{\diff\nu}{\diff\theta}=D^{-1}P_{1}\left\{ \frac{1}{n}\sum_{i=1}^{n}\omega_{i}^{r_{1}}\frac{\diff g\left(X_{i},\theta\right)}{\diff\theta},\frac{1}{n}\sum_{i=1}^{n}\omega_{i}^{r_{2}}g\left(X_{i},\theta\right),\ldots,\nu,\mu\right\} .
\]
Moreover, 
\begin{eqnarray*}
\frac{\diff\tilde{l}^{\mathrm{EL}}\left(\theta\right)}{\diff\theta} & = & \sum_{i=1}^{n}\frac{1}{1+\nu g\left(X_{i},\theta\right)}\left\{ \frac{\diff\nu}{\diff\theta}g\left(X_{i},\theta\right)+\nu\frac{\diff g\left(X_{i},\theta\right)}{\diff\theta}\right\} \\
 & = & \frac{\diff\nu}{\diff\theta}\frac{1}{n}\sum_{i=1}^{n}\omega_{i}\left(\theta\right)g\left(X_{i},\theta\right)+\nu\frac{1}{n}\sum_{i=1}^{n}\omega_{i}\left(\theta\right)\frac{\diff g\left(X_{i},\theta\right)}{\diff\theta},
\end{eqnarray*}
\[
\frac{\diff\tilde{l}^{\mathrm{ET}}\left(\theta\right)}{\diff\theta}=-\frac{\diff\nu}{\diff\theta}\frac{1}{n}\sum_{i=1}^{n}g\left(X_{i},\theta\right)-\nu\frac{1}{n}\sum_{i=1}^{n}\frac{\diff g\left(X_{i},\theta\right)}{\diff\theta}+\nu\sum_{i=1}^{n}\hat{w}_{i}\left(\theta\right)\frac{\diff g\left(X_{i},\theta\right)}{\diff\theta},
\]
\begin{eqnarray*}
\frac{\diff\tilde{l}^{\mathrm{CR}}\left(\theta\right)}{\diff\theta} & = & -\frac{1}{\lambda+1}\sum_{i=1}^{n}\omega_{i}^{\lambda+1}\left\{ \frac{\diff\mu}{\diff\theta}+\frac{\diff\nu}{\diff\theta}g\left(X_{i},\theta\right)+\nu\frac{\diff g\left(X_{i},\theta\right)}{\diff\theta}\right\} \\
 & = & -\frac{1}{\lambda+1}\left\{ \frac{\diff\mu}{\diff\theta}\frac{1}{n}\sum_{i=1}^{n}\omega_{i}^{\lambda+1}+\frac{\diff\nu}{\diff\theta}\frac{1}{n}\sum_{i=1}^{n}\omega_{i}^{\lambda+1}g\left(X_{i},\theta\right)+\nu\frac{1}{n}\sum_{i=1}^{n}\omega_{i}^{\lambda+1}\frac{\diff g\left(X_{i},\theta\right)}{\diff\theta}\right\} .
\end{eqnarray*}
So for $k=1$, the lemma holds. Assume for $k=n$, the lemma holds. Then 
for $k=n+1$, 
\begin{eqnarray*}
\frac{\diff^{n+1}\tilde{l}\left(\theta\right)}{\diff\theta^{n+1}} & = & \frac{\diff}{\diff\theta_{}}\frac{\diff^{k}\tilde{l}\left(\theta\right)}{\diff\theta}=-r_{k}D^{-r_{k}-1}\left(\frac{\diff D}{\diff\theta}\sum_{i=1}^{k_{n}}\frac{\diff P_{k}}{\diff M_{i}}\frac{\diff M_{i}}{\diff\theta}+\frac{\diff P_{k}}{\diff\nu}\frac{\diff\nu}{\diff\theta}+\frac{\diff P_{k}}{\diff\mu}\frac{\diff\mu}{\diff\theta}\right)
\end{eqnarray*}
The partial derivative of $P_{k}$ is still a polynomial. Also, $D$
itself is a polynomial in $n^{-1}\sum_{i=1}^{n}\omega_{i}^{r}g\left(X_{i},\theta\right)^2$,
$\mu$, and $\nu$. Next 
\begin{eqnarray*}
\frac{\diff M_{i}}{\diff\theta} & = & \frac{1}{n}\sum_{i=1}^{n}\left\{ r\omega_{i}^{r-1}\frac{\diff\omega_{i}}{\diff\theta}\prod_{l}\frac{\diff^{l}g\left(X_{i},\theta\right)}{\diff\theta^{l}}+\omega_{i}^{r}\sum_{l_{j}}\prod_{l\neq l_{j}}\frac{\diff^{l}g\left(X_{i},\theta\right)}{\diff\theta^{l}}\frac{\diff^{l_{j}+1}g\left(X_{i},\theta\right)}{\diff\theta^{l_{j}+1}}\right\} \\
 & = & \frac{r}{n}\sum_{i=1}^{n}\omega_{i}^{r-1}\prod_{l}\frac{\diff^{l}g\left(X_{i},\theta\right)}{\diff\theta^{l}}\frac{\diff\omega_{i}}{\diff\theta}+\sum_{l_{j}}\frac{1}{n}\sum_{i=1}^{n}\omega_{i}^{r}\prod_{l\neq l_{j}}\frac{\diff^{l}g\left(X_{i},\theta\right)}{\diff\theta^{l}}\frac{\diff^{l_{j}+1}g\left(X_{i},\theta\right)}{\diff\theta^{l_{j}+1}}.
\end{eqnarray*}
Similar to the calculation of the first order derivative of empirical
log likelihood, we know the $\diff\omega_{i}/\diff\theta$ are polynomials
involving terms like $M_{i}$. So for $k=n+1$, the higher order derivatives
of empirical log likelihood are of a similar form. Hence, by mathematical
induction, the lemma holds for all $n$. 
\end{proof}
By Lemma \ref{lem:control-higher-order-derivative-l}, the higher-order derivatives of log empirical likelihood are rational functions of the sample moments of higher-order derivatives of  $g$. We can anticipate
that higher order derivatives of log empirical likelihood can be bounded
in a small neighborhood of the true parameter when sample size is
large , provided the population moments of higher-order derivatives of function $g$ are finite. This we prove in the following lemma .


\begin{lemma}
\label{lem:bounded-high-order-der} Under Assumptions 3 , 4 and 5 ,
there exist constants $\delta_{2}$, $C_{3}$ and $N_{4}$ such that
for any $\left|b\left(\theta-\tilde{\theta}\right)\right|\le\delta_{2}$
and $n>N_{4}$, any $j=1,\ldots,k$,%
\begin{comment}
need add consistency conditions for M-Estimator
\end{comment}
{} 
\begin{equation}
\left|\frac{\diff^{j}\tilde{l}\left(\theta\right)}{\diff\theta^{j}}\right|\le C_{3}.\label{eq:bounded-higher-order-derivatives}
\end{equation}
\end{lemma}
\begin{proof}
All $\omega_{i}$ in Lemma \ref{lem:control-higher-order-derivative-l}
are equal to $1$ when evaluated at $\tilde{\theta}$. Under the assumption
of finite moments, by strong law of large numbers, and strong consistency
of the $M$-estimator $\tilde{\theta}$, we have 
\[
M_{j}=\frac{1}{n}\sum_{i=1}^{n}\prod_{l}\frac{\diff^{l}g\left(X_{i},\tilde{\theta}\right)}{\diff\theta^{l}}\rightarrow E\left\{ \prod_{l}\frac{\diff^{l}g\left(X_{1},\theta_{0}\right)}{\diff\theta^{l}}\right\} <\infty,\ascv.
\]
 By Lemma 3, the higher order
derivatives are continuous functions. Hence for any small number $\varepsilon_{2}>0$,
there exists a constant $\delta_{2}$ such that whenever $\left|b\left(\theta-\tilde{\theta}\right)\right|\le\delta_{2}$,
\[
\left|\frac{\diff^{j}\tilde{l}\left(\theta\right)}{\diff\theta^{j}}-\frac{\diff^{j}\tilde{l}\left(\tilde{\theta}\right)}{\diff\theta^{j}}\right|<\varepsilon_{2}.
\]
By Lemma \ref{lem:control-higher-order-derivative-l}, there exists
a constant $N_{4}$, such that whenever $n>N_{4}$. 
\[
\left|\frac{\diff^{j}\tilde{l}\left(\theta\right)}{\diff\theta^{j}}-\frac{P_{k}\left[E\left\{ \prod_{l}\diff^{l}g\left(X_{1},\theta_{0}\right)/\diff\theta^{l}\right\} ,\ldots,0,1\right]}{D^{r_{k}}}\right|<\varepsilon_{2}.
\]
By assumption, all the moments are bounded when $k\le K+3$.
Then there exist a constant $C_{3}$, such that 
\[
D^{-r_{k}}P_{k}\left[E\left\{ \prod_{l}\frac{\diff^{l}g\left(X_{1},\theta_{0}\right)}{\diff\theta^{l}}\right\} ,\ldots,0,1\right]\le C_{3},
\]
 %
\begin{comment}
change the length
\end{comment}
 which leads to (\ref{eq:bounded-higher-order-derivatives}). 
\end{proof}

\section{Expansion Near The M-Estimator}
\begin{lemma}
\label{lem:near-mean-2nd-order-bound-1}.  Under  Assumptions 1 and 2,there exists a $\delta_{3}>0$,
such that 
\[
\sum_{i=1}^{n}\log\hat{w}_{i}\left(\theta\right)-\sum_{i=1}^{n}\log\hat{w}_{i}\left(\tilde{\theta}\right)\le-\frac{1}{4}y^{2},
\]
for any $\theta\in\left\{ \theta:\left|b\left(\theta-\tilde{\theta}\right)\right|<\delta_{3}\right\} \cap H$
. \end{lemma}
\begin{proof}
By Taylor expansion, 
\begin{eqnarray*}
\frac{1}{n}\left\{ \sum_{i=1}^{n}\log\hat{w}_{i}\left(\theta\right)-\sum_{i=1}^{n}\log\hat{w}_{i}\left(\tilde{\theta}\right)\right\}  & = & \frac{\diff\tilde{l}\left(\tilde{\theta}\right)}{\diff\theta}\left(\theta-\tilde{\theta}\right)+\frac{1}{2}\frac{\diff^{2}\tilde{l}\left(\theta^{*}\right)}{\diff\theta^{2}}\left(\theta-\tilde{\theta}\right)^{2},
\end{eqnarray*}
where $\left|\theta^{*}-\tilde{\theta}\right|\le\left|\theta-\tilde{\theta}\right|$.
By Lemma 4, the first term in above equation
is zero. By Lemma \ref{lem:control-higher-order-derivative-l}, we
know that $\diff^{2}\tilde{l}\left(\theta\right)/\diff\theta^{2}$
is a continuous function in $\theta$. Thus there exists a $\delta_{3}$,
such that for any $\left|b\left(\theta^{*}-\tilde{\theta}\right)\right|\le\left|b\left(\theta-\tilde{\theta}\right)\right|<\delta_{3}$,
\[
\left|\frac{\diff^{2}\hat{l}\left(\theta^{*}\right)}{\diff\theta^{2}}\left(\theta-\tilde{\theta}\right)^{2}+\left|b\left(\theta-\tilde{\theta}\right)\right|^{2}\right|<\frac{1}{2}\left|b\left(\theta-\tilde{\theta}\right)\right|^{2}.
\]
Hence $\diff^{2}\tilde{l}\left(\theta^{*}\right)/\diff\theta^{2}\left(\theta-\tilde{\theta}\right)^{2}<-\left|b\left(\theta-\tilde{\theta}\right)\right|^{2}/2$,
 so that, 
\[
\sum_{i=1}^{n}\log\hat{w}_{i}\left(\theta\right)-\sum_{i=1}^{n}\log\hat{w}_{i}\left(\tilde{\theta}\right)<\frac{1}{2}\times\frac{1}{2}\left|\sqrt{n}b\left(\theta-\tilde{\theta}\right)\right|^{2}=\frac{1}{4}y^{2}.
\]
\end{proof}
The next lemma plays a key role in expanding the posterior, and can be interpreted as an empirical likelihood version of the Edgeworth expansion. 
\begin{lemma}
\label{lem:central-expansion-lik} Under  Assumptions 1 , 3 , 4 and 5 , then there exist $\delta_{4}$, $M_{3}$
and $N_{5}$, such that 
\[
\left|\int_{-\sqrt{n}\delta_{4}}^{\sqrt{n}\delta_{4}}\exp\left\{ -\frac{1}{2}y^{2}+\sum_{k=3}^{K+3}a_{kn}\left(\frac{y}{b}\right)^{k}n^{-\left(k-2\right)/2}\right\} -\prod_{i=1}^{n}\frac{\hat{w}_{i}\left(\theta\right)}{\hat{w}_{i}\left(\tilde{\theta}\right)}\diff y\right|\le M_{3}n^{-\left(K+2\right)/2},\:\ascv.
\]
\end{lemma}
\begin{proof}
Let $\delta_{4}\le\min\left(\delta_{2},\delta_{3}\right)$ in Lemma
\ref{lem:bounded-high-order-der} and Lemma \ref{lem:near-mean-2nd-order-bound-1}
\begin{eqnarray*}
 &  & \int_{-\sqrt{n}\delta_{4}}^{\sqrt{n}\delta_{4}}\exp\left\{ -\frac{1}{2}y^{2}+\sum_{k=3}^{K+3}a_{kn}\left(\frac{y}{b}\right)^{k}n^{-\left(k-2\right)/2}\right\} -\prod_{i=1}^{n}\frac{\hat{w}_{i}\left(\theta\right)}{\hat{w}_{i}\left(\tilde{\theta}\right)}\diff y\\
 & = & \int_{-\sqrt{n}\delta_{4}}^{\sqrt{n}\delta_{4}}\exp\left\{ \sum_{i=1}^{n}\log\hat{w}_{i}\left(\theta\right)-\sum_{i=1}^{n}\log\hat{w}_{i}\left(\tilde{\theta}\right)\right\} \\
 &  & \left[\exp\left\{ -\frac{1}{2}y^2+\sum_{k=3}^{K+3}a_{kn}\left(\frac{y}{b}\right)^{k}n^{-\left(k-2\right)/2}-\sum_{i=1}^{n}\log\hat{w}_{i}\left(\theta\right)+\sum_{i=1}^{n}\log\hat{w}_{i}\left(\tilde{\theta}\right)\right\} -1\right]\diff y.
\end{eqnarray*}
By %
Lemma \ref{lem:near-mean-2nd-order-bound-1}, %
Lemma \ref{lem:bounded-high-order-der} and Taylor expansion, the
above equation is bounded by 
\begin{eqnarray}
 &  & \int_{-\sqrt{n}\delta_{4}}^{\sqrt{n}\delta_{4}}\exp\left(-\frac{y^{2}}{4}\right)\left|\exp\left\{ -a_{K+4,n}\left(\theta^{*}\right)\left(\frac{y}{b}\right)^{K+4}n^{-\left(K+2\right)/2}\right\} -1\right|\diff y\nonumber \\
 & \le & \int_{-\sqrt{n}\delta_{4}}^{\sqrt{n}\delta_{4}}\exp\left(-\frac{y^{2}}{4}\right)\left|\exp\left\{ -C_{1}\left(\frac{y}{b}\right)^{K+4}n^{-\left(K+2\right)/2}\right\} -1\right|\diff y.\label{eq:bounded-central-int-likratio}
\end{eqnarray}
where $\left|\theta^{*}-\tilde{\theta}\right|\le\left|\theta-\tilde{\theta}\right|<\delta_{4}$,
and $C_{4}$ is some constant dependent on $\delta_{4}$, $N_{5}$
and $a_{K+4,n}\left(\tilde{\theta}\right)$. For sufficiently large
$n$, and sufficiently small $\delta_{4}$, $a_{K+4,n}\left(\theta^{*}\right)$
is very close to $a_{K+4,n}\left(\tilde{\theta}\right)$, and by %
Lemma \ref{lem:bounded-high-order-der}, $a_{K+4,n}\left(\tilde{\theta}\right)$
is finite. Hence, for very large $n$,  $\exp\left\{ -C_{4}\left(y/b\right)^{K+4}n^{-\left(k+2\right)/2}\right\} -1$
does not change sign on either $\left[-\sqrt{n}\delta_{4},0\right]$
and $\left[0,\sqrt{n}\delta_{4}\right]$. So without loss of generality,
we assume $\exp\left\{ -C_{4}\left(y/b\right)^{K+4}n^{-\left(K+2\right)/2}\right\} -1\ge0$
on $\left[-\sqrt{n}\delta_{4},\sqrt{n}\delta_{4}\right]$. With $t=\sqrt{n}$,%
\begin{comment}
unique the introduction of y
\end{comment}
{} and $t\in\mathbb{R}^{+}$, the last term in (\ref{eq:bounded-central-int-likratio})
can be written as 
\[
\int_{-\delta_{4}t}^{\delta_{4}t}\exp\left(-\frac{y^{2}}{4}\right)\left\{ \exp\left(-\frac{C_{4}}{b^{K+4}}y^{K+4}t^{-K-2}\right)-1\right\} \diff y.
\]
If we can show that 
\[
\lim_{t\rightarrow+\infty}\frac{\int_{-\delta_{4}t}^{\delta_{4}t}\exp\left(-y^{2}/4\right)\left\{ \exp\left(-C_{4}y^{K+4}t^{-K-2}/b^{K+4}\right)-1\right\} \diff y}{t^{-K-2}}=C_{5},
\]
for some $C_{5}<\infty$, the lemma is proved. Take the derivative
with respect to $t$ in both the numerator and the denominator. In the denominator,
$\left(t^{-K-2}\right)'=-\left(K+2\right)t^{-K-3}.$ In the numerator,
\begin{eqnarray}
 &  & \frac{\diff}{\diff t}\int_{-\delta_{4}t}^{\delta_{4}t}\exp\left(-\frac{y^{2}}{4}\right)\left\{ \exp\left(-\frac{C_{4}}{b^{K+4}}y^{K+4}t^{-K-2}\right)-1\right\} \diff y\nonumber \\
 & = & \int_{-\delta_{4}t}^{\delta_{4}t}\exp\left(-\frac{y^{2}}{4}\right)\left(-\frac{C_{4}}{b^{K+4}}y^{K+4}\right)\left(-K-2\right)t^{-K-3}\exp\left(-\frac{C_{4}}{b^{K+4}}y^{K+4}t^{-K-2}\right)\diff y\nonumber \\
 &  & +\exp\left\{ -\frac{\left(\delta_{4}t\right)^{2}}{4}\right\} \left[\exp\left\{ -\frac{C_{4}}{b^{K+4}}\left(\delta_{4}t\right)^{K+4}t^{-K-2}\right\} -1\right]\delta_{4}-\exp\left\{ -\frac{\left(-\delta_{4}t\right)^{2}}{4}\right\} \nonumber \\
 &  & \left[\exp\left\{ -\frac{C_{4}}{b^{K+4}}\left(-\delta_{4}t\right)^{K+4}t^{-K-2}\right\} -1\right]\left(-\delta_{4}\right)\nonumber \\
 & = & \frac{\left(K+2\right)C_{4}}{b^{K+4}}t^{-K-3}\int_{-\delta_{4}t}^{\delta_{4}t}y^{K+4}\exp\left(-\frac{y^{2}}{4}-\frac{C_{4}}{b^{K+4}}y^{K+4}t^{-K-2}\right)\diff y\nonumber \\
 &  & +\delta_{4}\left[\exp\left\{ -\left(\frac{\delta_{4}^{2}}{4}-\frac{C_{4}}{b^{K+4}}\delta_{4}^{K+4}\right)t^{2}\right\} -\exp\left(-\frac{\delta_{4}^{2}t^{2}}{4}\right)\right]\label{eq:lhospital-diff-numerator}\\
 &  & +\delta_{4}\left(\exp\left[-\left\{ \frac{\delta_{4}^{2}}{4}-\frac{C_{1}}{b^{K+4}}\left(-\delta_{4}\right)^{K+4}\right\} t^{2}\right]-\exp\left(-\frac{\delta_{4}^{2}t^{2}}{4}\right)\right).\nonumber 
\end{eqnarray}
We choose $\delta_{4}$ sufficiently small such that 
\begin{equation}
\delta_{4}<\min\left(\sqrt[K+2]{\frac{b^{K+4}}{4\left|C_{4}\right|}},\delta_{2},\delta_{3}\right).\label{eq:choose-delta-2}
\end{equation}
Hence, 
\[
0<\frac{\delta_{4}^{2}}{4}-\frac{\left|C_{4}\right|}{b^{K+4}}\delta_{4}^{K+4}\le\frac{\delta_{4}^{2}}{4}-\frac{C_{4}}{b^{K+4}}\left(\pm\delta_{4}\right)^{K+4},
\]
and 
\[
\lim_{t\rightarrow+\infty}\frac{\delta_{4}\left[\exp\left\{ -\left(\delta_{4}^{2}/4-\left|C_{4}\right|\delta_{2}^{K+4}/b^{K+4}\right)t^{2}\right\} -\exp\left(-\delta_{4}^{2}t^{2}/4\right)\right]}{-\left(K+2\right)t^{-K-3}}=0.
\]
Hence, the last two terms in (\ref{eq:lhospital-diff-numerator})
tend to zero when $t\rightarrow+\infty$. Now we consider the ratio
\begin{eqnarray}
 &  & \frac{\left\{ \left(K+2\right)C_{4}/b^{K+4}\right\} t^{-K-3}\int_{-\delta_{4}t}^{\delta_{4}t}y^{K+4}\exp\left(-y^{2}/4-C_{4}y^{K+4}t^{-K-2}/b^{K+4}\right)\diff y}{-\left(K+2\right)t^{-K-3}}\nonumber \\
 & = & -\frac{C_{4}}{b^{K+4}}\int_{-\delta_{4}t}^{\delta_{4}t}y^{K+4}\exp\left(-\frac{y^{2}}{4}-\frac{C_{4}}{b^{K+4}}y^{K+4}t^{-K-2}\right)\diff y.\label{eq:lhopstal-diff-1st-ratio}
\end{eqnarray}
Since $\delta_{4}$ satisfies (\ref{eq:choose-delta-2}), (\ref{eq:lhopstal-diff-1st-ratio})
is bounded by 
\begin{eqnarray*}
 &  & \frac{\left|C_{4}\right|}{b^{K+4}}\int_{-\delta_{4}t}^{\delta_{4}t}\left|y\right|^{K+4}\exp\left\{ -\frac{y^{2}}{4}-\frac{\left|C_{4}\right|}{b^{K+4}}\left(\delta_{4}t\right)^{K+2}y^{2}t^{-K-2}\right\} \diff y\\
 & = & \frac{\left|C_{4}\right|}{b^{K+4}}\int_{-\delta_{4}t}^{\delta_{4}t}\left|y\right|^{K+4}\exp\left\{ -\left(\frac{\delta_{4}^{2}}{4}-\frac{\left|C_{4}\right|}{b^{K+4}}\delta_{4}^{K+4}\right)y^{2}\right\} \diff y\\
 & \rightarrow & \frac{\left|C_{4}\right|}{b^{K+4}}\sqrt{2\pi\left\{ 2\left(\frac{\delta_{4}^{2}}{4}-\frac{\left|C_{4}\right|}{b^{K+4}}\delta_{4}^{K+4}\right)\right\} ^{-1}}\left\{ 2\left(\frac{\delta_{4}^{2}}{4}-\frac{\left|C_{4}\right|}{b^{K+4}}\delta_{4}^{K+4}\right)\right\} ^{-\left(K+4\right)/2}\\ & &\frac{2^{\left(K+4\right)/2}\Gamma\left\{ \left(K+4+1\right)/2\right\} }{\sqrt{\pi}}<\infty.
\end{eqnarray*}
So by L'Hostiple's rule, we have 
\begin{eqnarray*}
 &  & \lim_{t\rightarrow+\infty}\frac{\int_{-\delta_{4}t}^{\delta_{4}t}\exp\left(-y^{2}/4\right)\left\{ \exp\left(-C_{4}y^{K+4}t^{-K-2}/b^{K+4}\right)-1\right\} \diff y}{t^{-K-2}}\\
 & = & \lim_{t\rightarrow+\infty}\frac{\left[\int_{-\delta_{4}t}^{\delta_{4}t}\exp\left(-y^{2}/4\right)\left\{ \exp\left(-C_{4}y^{K+4}t^{-K-2}/b^{K+4}\right)-1\right\} \diff y\right]'}{\left(t^{-K-2}\right)'}=C_{5}<\infty.
\end{eqnarray*}
\end{proof}
\begin{lemma}
\label{lem:central-expansion-post-prod}%
 Under  Assumptions 1 , 3 , 4 and 5 ,  there exists  $\delta_{4}>0$, and constants
$M_{4}$, $N_{6}$, such that 
\begin{equation}
\left|\int_{-\sqrt{n}\delta_{4}}^{\sqrt{n}\delta_{4}}\left[\exp\left\{ -\frac{1}{2}y^{2}+\sum_{k=3}^{K+3}a_{kn}\left(\frac{y}{b}\right)^{k}n^{-\left(k-2\right)/2}\right\} \rho_{K}\left(\theta\right)-\prod_{i=1}^{n}\frac{\hat{w}_{i}\left(\theta\right)}{\hat{w}_{i}\left(\tilde{\theta}\right)}\rho\left(\theta\right)\right]\diff y\right|\le M_{4}n^{-\frac{1}{2}\left(K+1\right)},\:\ascv.\label{eq:central-exp-post}
\end{equation}
\end{lemma}
\begin{proof}
Use $\delta_{4}$ in Lemma \ref{lem:central-expansion-lik}, and apply
Taylor expansion of $\tilde{l}\left(\theta\right)$ around $\tilde{\theta}$.
Then for any $\tilde{\theta}-\delta_{4}/b\le\theta\le\tilde{\theta}+\delta_{4}/b$,
there exists a $\theta^{*}$ which satisfies $\left|b\left(\theta^{*}-\tilde{\theta}\right)\right|\le\left|b\left(\theta-\tilde{\theta}\right)\right|$.
This leads to 
\begin{eqnarray*}
\tilde{l}\left(\theta\right) & = & \tilde{l}\left(\tilde{\theta}\right)+\frac{\diff\tilde{l}\left(\tilde{\theta}\right)}{\diff\theta}\left(\theta-\tilde{\theta}\right)+\frac{1}{2}\frac{\diff^{2}\tilde{l}\left(\tilde{\theta}\right)}{\diff\theta^{2}}\left(\theta-\tilde{\theta}\right)^{2}+\sum_{k=3}^{K+3}a_{kn}\left(\tilde{\theta}\right)\left(\theta-\tilde{\theta}\right)^{k}\\
 &  & +\frac{1}{\left(K+4\right)!}\frac{\diff^{K+4}}{\diff\theta}\tilde{l}\left(\theta^{*}\right)\left(\theta-\tilde{\theta}\right)^{K+4}\\
 & = & \tilde{l}\left(\tilde{\theta}\right)-\frac{1}{2}y^{2}n^{-1}+\sum_{k=3}^{K+3}a_{kn}\left(\frac{y}{b}\right)^{k}n^{-k/2}+\frac{1}{\left(K+4\right)!}\frac{\diff^{K+4}}{\diff\theta}\tilde{l}\left(\theta^{*}\right)\left(\theta-\tilde{\theta}\right)^{K+4}.
\end{eqnarray*}
Now 
\begin{eqnarray*}
 &  & \left|\exp\left\{ -\frac{1}{2}y^{2}+\sum_{k=3}^{K+3}a_{kn}\left(\frac{y}{b}\right)^{k}n^{-\left(k-2\right)/2}\right\} \rho_{K}\left(\theta\right)-\prod_{i=1}^{n}\frac{\hat{w}_{i}\left(\theta\right)}{\hat{w}_{i}\left(\tilde{\theta}\right)}\rho\left(\theta\right)\right|\\
 &  & \le\left|\exp\left\{ -\frac{1}{2}y^{2}+\sum_{k=3}^{K+3}a_{kn}\left(\frac{y}{b}\right)^{k}n^{-\left(k-2\right)/2}\right\} \rho_{K}\left(\theta\right)-\prod_{i=1}^{n}\frac{\hat{w}_{i}\left(\theta\right)}{\hat{w}_{i}\left(\tilde{\theta}\right)}\rho_{K}\left(\theta\right)\right|\\
 &  & +\left|\prod_{i=1}^{n}\frac{\hat{w}_{i}\left(\theta\right)}{\hat{w}_{i}\left(\tilde{\theta}\right)}\rho_{K}\left(\theta\right)-\prod_{i=1}^{n}\frac{\hat{w}_{i}\left(\theta\right)}{\hat{w}_{i}\left(\tilde{\theta}\right)}\rho\left(\theta\right)\right|\\
 &  & \le\left|\rho_{K}\left(\theta\right)\right|\exp\left\{ n\tilde{l}\left(\theta\right)-n\tilde{l}\left(\tilde{\theta}\right)\right\} \left|\exp\left[n\left\{ \tilde{l}\left(\tilde{\theta}\right)-\frac{1}{2}y^{2}n^{-1}+\sum_{k=3}^{K+3}a_{kn}\left(\frac{y}{b}\right)^{k}n^{-k/2}-\tilde{l}\left(\theta\right)\right\} \right]-1\right|\\
 &  & +\exp\left\{ n\tilde{l}\left(\theta\right)-n\tilde{l}\left(\tilde{\theta}\right)\right\} \left|\rho_{K}\left(\theta\right)-\rho\left(\theta\right)\right|.
\end{eqnarray*}
 By Lemma \ref{lem:central-expansion-lik}, the first term in the
right hand side is bounded by 
\[
\left\{ \max_{\tilde{\theta}-\delta_{4}/b\le\theta\le\tilde{\theta}+\delta_{4}/b}\rho_{K}\left(\theta\right)\right\} M_{3}n^{-\left(K+2\right)/2}.
\]
By Lemma \ref{lem:near-mean-2nd-order-bound-1}, and Taylor expansion
of $\rho\left(\theta\right)$, the second term is bounded by 
\begin{eqnarray*}
 &  & \int_{-\sqrt{n}\delta_{4}}^{\sqrt{n}\delta_{4}}\exp\left(-\frac{y^{2}}{4}\right)\frac{n^{-\left(K+1\right)/2}}{\left(K+1\right)!}\rho^{K+1}\left(\theta^{*}\right)y^{K+1}\diff y\\
 & \le & \frac{1}{\left(K+1\right)!}\left\{ \max_{\tilde{\theta}-\delta_{4}/b\le\theta\le\tilde{\theta}+\delta_{4}/b}\rho^{K+1}\left(\theta^{*}\right)\right\} \left\{ \int_{-\sqrt{n}\delta_{4}}^{\sqrt{n}\delta_{4}}\exp\left(-\frac{y^{2}}{4}\right)y^{\left(K+1\right)/2}\diff y\right\} n^{-\left(K+1\right)/2}\\
 & = & \left[\frac{1}{\left(K+1\right)!}\left\{ \max_{\tilde{\theta}-\delta_{4}/b\le\theta\le\tilde{\theta}+\delta_{4}/b}\rho^{K+1}\left(\theta^{*}\right)\right\} \int_{0}^{\infty}\exp\left(-\frac{y^{2}}{4}\right)y^{\left(K+1\right)/2}\diff y\right]n^{-\left(K+1\right)/2}
\end{eqnarray*}
Hence (\ref{eq:central-exp-post}) holds. %
\begin{comment}
add some constant for the bound of rho
\end{comment}

\end{proof}

\section{Proof Of The Fundamental Theorem For Expansion}\label{app-proof-fun-thm}

We first intuitively derive $P_{K}\left(\xi,n\right)$ %
\begin{comment}
add expansion polynomial
\end{comment}
. First, we expand 
\begin{eqnarray*}
 &  & \exp\left\{ \sum_{k=3}^{K+3}a_{kn}\left(\frac{y}{b}\right)^{k}n^{-\left(k-2\right)/2}\right\} \\
 & = & \sum_{i=0}^{K+1}\frac{1}{i!}\left\{ \sum_{k=3}^{K+3}a_{kn}\left(\frac{y}{b}\right)^{k}n^{-\left(k-2\right)/2}\right\} ^{i}\\
 & = & 1+\sum_{i=1}^{K+1}\frac{1}{i!}\sum_{\sum_{u=3}^{K+3}m_{u,i}=i}\binom{i}{m_{3,i},\ldots,m_{K+3,i}}\prod_{u=3}^{K+3}\left(a_{un}\right)^{m_{u,i}}\left(\frac{y}{b}\right)^{\sum_{u=3}^{K+3}m_{u,i}u}n^{-\left\{ \sum_{u=3}^{K+3}m_{u,i}\left(u-2\right)\right\} /2}.
\end{eqnarray*}
Multiplying the above expression by $\rho_{K}$, 
\begin{eqnarray*}
 &  & \exp\left\{ \sum_{k=3}^{K+3}a_{kn}\left(\frac{y}{b}\right)^{k}n^{-\left(k-2\right)/2}\right\} \rho_{K}\left(\theta\right)\\
 & = & \left[1+\sum_{i=1}^{K+1}\frac{1}{i!}\sum_{\sum_{u=3}^{K+3}m_{u,i}=i}\binom{i}{m_{3,i},\ldots,m_{K+3,i}}\prod_{u=3}^{K+3}\left(a_{un}\right)^{m_{u,i}}\left(\frac{y}{b}\right)^{\sum_{u=3}^{K+3}m_{u,i}u}n^{-\left\{ \sum_{u=3}^{K+3}m_{u,i}\left(u-2\right)\right\} /2}\right]\\
 &  & \times\left\{ \rho\left(\tilde{\theta}\right)+\sum_{j=1}^{K}\frac{1}{j!}\rho^{\left(j\right)}\left(\frac{y}{b}\right)^{j}n^{-j/2}\right\} \\
 & = & \rho\left(\tilde{\theta}\right)+\sum_{j=1}^{K}\frac{1}{j!}\rho^{\left(j\right)}\left(\frac{y}{b}\right)^{j}n^{-j/2}+\rho\left(\tilde{\theta}\right)\sum_{i=1}^{K+1}\frac{1}{i!}\\
 &  & \sum_{\sum_{u=3}^{K+3}m_{u,i}=i}\binom{i}{m_{3,i},\ldots,m_{K+3,i}}\prod_{u=3}^{K+3}\left(a_{un}\right)^{m_{u,i}}\left(\frac{y}{b}\right)^{\sum_{u=3}^{K+3}m_{u,i}u}n^{-\left\{ \sum_{u=3}^{K+3}m_{u,i}\left(u-2\right)\right\} /2}\\
 &  & +\left[\sum_{i=1}^{K+1}\frac{1}{i!}\sum_{\sum_{u=3}^{K+3}m_{u,i}=i}\binom{i}{m_{3,i},\ldots,m_{K+3,i}}\prod_{u=3}^{K+3}\left(a_{un}\right)^{m_{u,i}}\left(\frac{y}{b}\right)^{\sum_{u=3}^{K+3}m_{u,i}u}n^{-\left\{ \sum_{u=3}^{K+3}m_{u,i}\left(u-2\right)\right\} /2}\right]\\
 &  & \times\sum_{j=1}^{K}\frac{1}{j!}\rho^{\left(j\right)}\left(\frac{y}{b}\right)^{j}n^{-j/2}.
\end{eqnarray*}
For the third term in the right hand side of the above equation, we
change the summation index. Let $\sum_{u=3}^{K+3}m_{u,i}\left(u-2\right)=h$.
For any $\sum_{u=3}^{K+3}m_{u,i}=i$, $i\le h\le i\left(K+1\right)$,
$h/\left(K+1\right)\le i\le h$. Thus the third term in the summation can
be rearranged as 
\[
\sum_{h=1}^{\left(K+1\right)^{2}}\left\{ \rho\left(\tilde{\theta}\right)\sum_{i=\left\lceil h/\left(K+1\right)\right\rceil }^{h}\frac{1}{i!}\sum_{I_{i,h}}\binom{i}{m_{3,i},\ldots,m_{K+3,i}}\prod_{u=3}^{K+3}\left(a_{un}\right)^{m_{u,i}}\left(\frac{y}{b}\right)^{\sum_{u=3}^{K+3}m_{u,i}u}\right\} n^{-h/2}.
\]
Similarly for the fourth term, let $\sum_{u=3}^{K+3}m_{u,i}\left(u-2\right)+j=h$.
Then the summation can be rearranged as 
\[
\sum_{h=2}^{\left(K+1\right)^{2}+K}\left\{ \sum_{j=1}^{h-1}\frac{1}{j!}\rho^{\left(j\right)}\left(\frac{y}{b}\right)^{j}\sum_{i=\lceil \left(h-j\right)/\left(K+1\right) \rceil}^{h-j}\frac{1}{i!}\sum_{I_{i,h-j}}\binom{i}{m_{3,i},\ldots,m_{K+3,i}}\prod_{u=3}^{K+3}\left(a_{un}\right)^{m_{u,i}}\left(\frac{y}{b}\right)^{\sum_{u=3}^{K+3}m_{u,i}u}\right\} n^{-h/2}.
\]
We collect  same order terms of $n$, and denote the summation of
all the terms with order higher than $K$ to be $R_{K}\left(Y\right)$.
Then we get the product as 
\begin{eqnarray*}
 &  & \rho\left(\tilde{\theta}\right)+\left\{ \rho'\left(\frac{y}{b}\right)+\rho\left(\tilde{\theta}\right)a_{3n}\left(\frac{y}{b}\right)^{3}\right\} n^{-1/2}\\
 &  & +\sum_{h=2}^{K}\Bigg\{ \frac{1}{h!}\rho^{\left(h\right)}\left(\frac{y}{b}\right)^{h}+\sum_{j=0}^{h-1}\frac{1}{j!}\rho^{\left(j\right)}\left(\frac{y}{b}\right)^{j}\\
 &&\times\sum_{i=\left\lceil \left(h-j\right)/\left(K+1\right)\right\rceil }^{h-j}\frac{1}{i!}\sum_{I_{i,h-j}}\binom{i}{m_{3,i},\ldots,m_{K+3,i}}\prod_{u=3}^{K+3}\left(a_{un}\right)^{m_{u,i}}\left(\frac{y}{b}\right)^{\sum_{u=3}^{K+3}m_{u,i}u}\Bigg\} n^{-h/2}+R_{K}\left(y\right).
\end{eqnarray*}
 Integrating any Borel set $\left(Y_{\left(1\right)},\xi\right]$%
\begin{comment}
change into x1 to xi
\end{comment}
, we get the polynomial $P_{K}\left(\xi,n\right)$. Now we prove Theorem
1.%


. 
\begin{proof}
Let $A_{1}=\left\{ \left|y\right|\ge\delta_{4}\sqrt{n}\right\} \cap\left(Y_{\left(1\right)},\xi\right)$
and $A_{2}=\left\{ \left|y\right|<\delta_{4}\sqrt{n}\right\} \cap\left(Y_{\left(1\right)},\xi\right)$,
where $\delta_{4}$ is in Lemma \ref{lem:central-expansion-post-prod}.
Then 
\begin{eqnarray*}
 &  & \left|\int_{Y_{\left(1\right)}}^{\xi}\prod_{i=1}^{n}\frac{\hat{w}_{i}\left(\tilde{\theta}+y/\sqrt{n}b\right)}{\hat{w}_{i}\left(\tilde{\theta}\right)}\rho\left(\tilde{\theta}+\frac{y}{\sqrt{n}b}\right)\diff y-P_{K}\left(\xi,n\right)\right|\\
 & = & \left|\int_{Y_{\left(1\right)}}^{\xi}\left\{ \prod_{i=1}^{n}\frac{\hat{w}_{i}\left(\tilde{\theta}+y/\sqrt{n}b\right)}{\hat{w}_{i}\left(\tilde{\theta}\right)}\rho\left(\tilde{\theta}+\frac{y}{\sqrt{n}b}\right)-\exp\left(-\frac{y^{2}}{2}\right)\sum_{h=0}^{K}\alpha_{h}\left(y,n\right)n^{-h/2}\right\} \diff y\right|\\
 & \le & \left|\int_{A_{1}}\left\{ \prod_{i=1}^{n}\frac{\hat{w}_{i}\left(\tilde{\theta}+y/\sqrt{n}b\right)}{\hat{w}_{i}\left(\tilde{\theta}\right)}\rho\left(\tilde{\theta}+\frac{y}{\sqrt{n}b}\right)-\exp\left(-\frac{y^{2}}{2}\right)\sum_{h=0}^{K}\alpha_{h}\left(y,n\right)n^{-h/2}\right\} \diff y\right|\\
 &  & +\left|\int_{A_{2}}\left\{ \prod_{i=1}^{n}\frac{\hat{w}_{i}\left(\tilde{\theta}+y/\sqrt{n}b\right)}{\hat{w}_{i}\left(\tilde{\theta}\right)}\rho\left(\tilde{\theta}+\frac{y}{\sqrt{n}b}\right)-\exp\left(-\frac{y^{2}}{2}\right)\sum_{h=0}^{K}\alpha_{h}\left(y,n\right)n^{-h/2}\right\} \diff y\right|.
\end{eqnarray*}


For the first term%
, by Lemma \ref{lemma:exponential-decay-tail}, we have
\begin{eqnarray*}
 &  & \left|\int_{A_{1}}\prod_{i=1}^{n}\frac{\hat{w}_{i}\left(\tilde{\theta}+y/\sqrt{n}b\right)}{\hat{w}_{i}\left(\tilde{\theta}\right)}\rho\left(\tilde{\theta}+\frac{y}{\sqrt{n}b}\right)-\exp\left(-\frac{y^{2}}{2}\right)\sum_{h=0}^{K}\alpha_{h}\left(y,n\right)n^{-h/2}\diff y\right|\\
 & \le & \int_{A_{1}}\exp\left\{ n\hat{l}\left(\tilde{\theta}+\frac{y}{\sqrt{n}b}\right)-n\hat{l}\left(\tilde{\theta}\right)\right\} \rho\left(\tilde{\theta}+\frac{y}{\sqrt{n}b}\right)\diff y\\
 &  & +\left|\int_{A_{1}}\exp\left(-\frac{y^{2}}{4}\right)\sum_{h=0}^{K}\alpha_{h}\left(y,n\right)n^{-h/2}\diff y\right|\\
 & \le & \exp\left(-n\varepsilon\right)\int_{A_{1}}\rho\left(\tilde{\theta}+\frac{y}{\sqrt{n}b}\right)\diff y\\
 &  & +\exp\left(-\frac{\delta_{4}^{2}n}{4}\right)\left|\int_{A_{1}}\exp\left(-\frac{y^{2}}{4}\right)\sum_{h=0}^{K}\alpha_{h}\left(y,n\right)n^{-h/2}\diff y\right|\\
 & \le & \exp\left(-n\varepsilon\right)\int_{\mathbb{R}}\rho\left(\tilde{\theta}+\frac{y}{\sqrt{n}b}\right)\diff y+\exp\left(-n\frac{\delta_{4}^{2}}{4}\right)\sum_{h=0}^{K}\left\{ \int_{Y_{\left(1\right)}}^{Y_{\left(n\right)}}\exp\left(-\frac{y^{2}}{4}\right)\left|\alpha_{h}\left(y,n\right)\right|\diff y\right\} n^{-h/2}.
\end{eqnarray*}
The above terms are exponentially decreasing with respect to $n$.
Hence there exist  $N_{1}$, and $M_{5}$, such that for any $n\ge N_{1}$,
\[
\left|\int_{A_{1}}\prod_{i=1}^{n}\frac{\hat{w}_{i}\left(\tilde{\theta}+y/\sqrt{n}b\right)}{\hat{w}_{i}\left(\tilde{\theta}\right)}\rho\left(\tilde{\theta}+\frac{y}{\sqrt{n}b}\right)-\exp\left(-\frac{y^{2}}{2}\right)\sum_{h=0}^{K}\alpha_{h}\left(y,n\right)n^{-h/2}\diff y\right|\le M_{5}n^{-\left(K+1\right)/2}.
\]


For the second term, by Lemma \ref{lem:central-expansion-post-prod},
we have 
\begin{eqnarray*}
 &  & \left|\int_{A_{2}}\prod_{i=1}^{n}\frac{\hat{w}_{i}\left(\tilde{\theta}+y/\sqrt{n}b\right)}{\hat{w}_{i}\left(\tilde{\theta}\right)}\rho\left(\tilde{\theta}+\frac{y}{\sqrt{n}b}\right)-\exp\left(-\frac{y^{2}}{2}\right)\sum_{h=0}^{K}\alpha_{h}\left(y,n\right)n^{-h/2}\diff y\right|\\
 & \le & \left|\int_{A_{2}}\prod_{i=1}^{n}\frac{\hat{w}_{i}\left(\tilde{\theta}+y/\sqrt{n}b\right)}{\hat{w}_{i}\left(\tilde{\theta}\right)}\rho\left(\tilde{\theta}+\frac{y}{\sqrt{n}b}\right)-\exp\left\{ -\frac{1}{2}y^{2}+\sum_{k=3}^{K+3}a_{kn}\left(\tilde{\theta}\right)\left(\frac{y}{b}\right)^{k}n^{-\left(K-2\right)/2}\right\} \rho_{K}\left(\theta\right)\diff y\right|\\
 &  & +\Bigg|\int_{A_{2}}\exp\left\{ -\frac{1}{2}y^{2}+\sum_{k=3}^{K+3}a_{kn}\left(\tilde{\theta}\right)\left(\frac{y}{b}\right)^{k}n^{-\left(K-2\right)/2}\right\} \rho_{K}\left(\tilde{\theta}+\frac{y}{\sqrt{n}b}\right)\\
 & & -\exp\left(-\frac{y^{2}}{2}\right)\sum_{h=0}^{K}\alpha_{h}\left(y,n\right)n^{-h/2}\diff y\Bigg|\\
 & \le & M_{4}n^{-\left(K+1\right)/2}\\
 &  & +\Bigg|\int_{A_{2}}\exp\left\{ -\frac{1}{2}y^{2}+\sum_{k=3}^{K+3}a_{kn}\left(\tilde{\theta}\right)\left(\frac{y}{b}\right)^{k}n^{-\left(K-2\right)/2}\right\} \rho_{K}\left(\tilde{\theta}+\frac{y}{\sqrt{n}b}\right)\\
 & &-\exp\left(-\frac{y^{2}}{2}\right)\sum_{h=0}^{K}\alpha_{h}\left(y,n\right)n^{-h/2}\diff y\Bigg|.
\end{eqnarray*}
For the second term in the right hand side, we add and subtract $R_{K}\left(y\right)$
in integrand, and by Taylor expansion, 
\begin{eqnarray*}
 &  & \left|\int_{A_{2}}\exp\left\{ -\frac{1}{2}y^{2}+\sum_{k=3}^{K+3}a_{kn}\left(\tilde{\theta}\right)\left(\frac{y}{b}\right)^{k}n^{-\left(K-2\right)/2}\right\} \rho_{K}\left(\tilde{\theta}+\frac{y}{\sqrt{n}b}\right)-\exp\left(-\frac{y^{2}}{2}\right)\sum_{h=0}^{K}\alpha_{h}\left(y,n\right)n^{-h/2}\diff y\right|\\
 & \le & \left|\int_{A_{2}}\exp\left(-\frac{y^{2}}{2}\right)\rho_{K}\left(\theta\right)\left[\exp\left\{ \sum_{k=3}^{K+3}a_{kn}\left(\tilde{\theta}\right)\left(\frac{y}{b}\right)^{k}n^{-\left(K-2\right)/2}\right\} -\sum_{i=0}^{K+1}\frac{1}{i!}\left\{ \sum_{k=3}^{K+3}a_{kn}\left(\tilde{\theta}\right)\left(\frac{y}{b}\right)^{k}n^{-\left(K-2\right)/2}\right\} ^{i}\right]\diff y\right|\\
 &  & +\left|\int_{A_{2}}\exp\left(-\frac{y^{2}}{2}\right)R_{K}\left(y\right)\diff n^{-1/2}y\right|\\
 & = & \left|\int_{A_{2}}\exp\left(-\frac{y^{2}}{2}\right)\rho_{K}\left(\theta\right)\frac{1}{\left(K+2\right)!}\exp\left(L\right)\left\{ \sum_{k=3}^{K+3}a_{kn}\left(\tilde{\theta}\right)\left(\frac{y}{b}\right)^{k}n^{-\left(K-2\right)/2}\right\} ^{K+2}\diff y\right|\\
 &  & +\left|\int_{A_{2}}\exp\left(-\frac{y^{2}}{2}\right)R_{K}\left(y\right)\diff y\right|,
\end{eqnarray*}
where $\left|L\right|\le\left|\sum_{k=3}^{K+3}a_{kn}\left(\tilde{\theta}\right)\left(y/b\right)^{k}n^{-\left(K-2\right)/2}\right|$.
We know that $R_{K}\left(y\right)$ is a polynomial with order $n^{-\left(K+1\right)/2}$.
So there exists an $M_{6}$, such that 
\[
\left|\int_{A_{2}}\exp\left(-\frac{y^{2}}{2}\right)R_{K}\left(y\right)\diff y\right|\le M_{6}n^{-\left(K+1\right)/2}.
\]
For the first term, 

\begin{eqnarray}
 &  & \left|\int_{A_{2}}\exp\left(-\frac{y^{2}}{2}\right)\rho_{K}\left(\theta\right)\frac{1}{\left(K+2\right)!}\exp\left(L\right)\left\{ \sum_{k=3}^{K+3}a_{kn}\left(\tilde{\theta}\right)\left(\frac{y}{b}\right)^{k}n^{-\left(K-2\right)/2}\right\} ^{K+2}\diff n^{-1/2}Y\right|\label{eq:1}\\
 & \le & \frac{1}{\left(K+2\right)!}\left|\int_{A_{2}}\exp\left(-\frac{y^{2}}{2}\right)\rho_{K}\left(\theta\right)\exp\left\{ \left|\sum_{k=3}^{K+3}a_{kn}\left(\tilde{\theta}\right)\left(\frac{y}{b}\right)^{k}n^{-\left(K-2\right)/2}\right|\right\} \left\{ \sum_{k=3}^{K+3}a_{kn}\left(\tilde{\theta}\right)\left(\frac{y}{b}\right)^{k}n^{-\left(K-2\right)/2}\right\} ^{K+2}\diff y\right|\nonumber \\
 & = & \frac{1}{\left(K+2\right)!}\Bigg|\int_{A_{2}}\rho_{K}\left(\theta\right)\exp\left\{ -\frac{y^{2}}{2}+\left|\sum_{k=3}^{K+3}a_{kn}\left(\tilde{\theta}\right)\left(\frac{y}{b}\right)^{k}n^{-\left(K-2\right)/2}\right|\right\} \nonumber \\
 &  & \left\{ \sum_{k=3}^{K+3}a_{kn}\left(\tilde{\theta}\right)\left(\frac{y}{b}\right)^{k}n^{-\left(K-2\right)/2}\right\} ^{K+2}\diff y\Bigg|.\nonumber 
\end{eqnarray}
We need $\delta_{4}$ sufficiently small, so that there exist $C_{6}$
and $C_{7}$, such that 
\begin{eqnarray*}
\frac{y^{2}}{2}-\left|\sum_{k=3}^{K+3}a_{kn}\left(\tilde{\theta}\right)\left(\frac{y}{b}\right)^{k}n^{-\left(K-2\right)/2}\right| & \ge & C_{6}y^{2},\\
\sum_{k=3}^{K+3}a_{kn}\left(\tilde{\theta}\right)\left(\frac{y}{b}\right)^{k}n^{-\left(K-2\right)/2} & \le & C_{7}y^{3}n^{-1/2}.
\end{eqnarray*}
 Hence, (\ref{eq:1}) is bounded by 
\begin{eqnarray*}
 &  & \frac{n^{K+2}}{\left(K+2\right)!}\left|\int_{A_{n}}\exp\left(-C_{6}y^{2}\right)\left(C_{7}y^{3}n^{-1/2}\right)^{K+2}\diff y\right|\\
 & \le & \frac{C_{7}^{K+2}}{\left(K+2\right)!}\left|\int_{Y_{\left(1\right)}}^{Y_{\left(n\right)}}\exp\left(-C_{6}y^{2}\right)y^{3\left(K+2\right)}n^{-\left(K+2\right)/2}\diff y\right|\\
 & \le & \frac{C_{7}^{K+2}}{\left(K+2\right)!}\left|\int_{Y_{\left(1\right)}}^{Y_{\left(n\right)}}\exp\left(-C_{6}y^{2}\right)y^{3\left(K+2\right)}\diff y\right|n^{-\left(K+2\right)/2}.
\end{eqnarray*}
Adding all the parts , we get the inequality in Theorem 1
\begin{comment}
add ref to main theorem
\end{comment}
.
\end{proof}

